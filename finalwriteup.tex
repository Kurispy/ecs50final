\documentclass[11pt]{article}

\setlength{\oddsidemargin}{0.0in}
\setlength{\evensidemargin}{0.0in}
\setlength{\topmargin}{-0.25in}
\setlength{\headheight}{0in}
\setlength{\headsep}{0in}
\setlength{\textwidth}{6.5in}
\setlength{\textheight}{9.25in}
\setlength{\parindent}{0in}
\setlength{\parskip}{2mm}

\begin{document}

1. There is no function prologue and epilogue in the optimized code. (In worker and crossout)

2. The optimized code uses a cpu specific instruction to perform a square root operation instead of a call to sqrt from the math library. (Line 65 in opt. code vs line 63 in normal code)

3. The optimized code uses more efficient addressing modes. (line 125-126 non op vs line 140 op code and line 131-132 non op vs line 144 op code)

% the $ delimiter marks the start and end of a mathematical expression

\begin{equation}
\theta = cos^{-1}(\frac{x}{z})  % note the need for braces around the -1
\end{equation}

% the parentheses in the above expression aren't quite big enough;
% instead, we could have the LaTeX code
% \begin{equation}
% \theta = cos^{-1} \left ( \frac{x}{z} \right)  
% \end{equation}
% try it yourself (don't forget to remove the % comment symbols first)

\end{document}  % required; the document ends here
